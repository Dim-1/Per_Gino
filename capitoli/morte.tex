\chapter{La paura della morte}

\begin{quotation}
	\small L’antica leggenda narra che il re Mida inseguì a lungo nella foresta il saggio Sileno \textit{(un satiro)}, seguace di Dioniso, senza prenderlo.
	Quando quello gli cadde infine tra le mani, il re domandò quale fosse la cosa migliore e più desiderabile per l’uomo.
	Rigido e immobile, il demone tace; finché, costretto dal re, esce da ultimo fra stridule risa in queste parole:
	
\textbf{	«Stirpe miserabile ed effimera, figlio del caso e della pena, perché mi costringi a dirti ciò che per te è vantaggiosissimo non sentire?
	Il meglio è per te assolutamente irraggiungibile: non essere nato, non essere, essere niente.
	Ma la cosa in secondo luogo migliore per te è morire presto.»}\footnote{ F. Nietzsche, La nascita della tragedia, Adelphi, Milano 2018}
\end{quotation}

