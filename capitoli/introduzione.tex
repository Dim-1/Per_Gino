\begin{quotation}
	\small L'uomo libero a nessuna cosa pensa meno che alla morte: e la sua saggezza è una meditazione della vita, non della morte - Baruch Spinoza, Ethica more geometrico demonstata
\end{quotation}

Quando leggerai questo piccolo libro, caro Gino mio, babbo non ci sarà più. E' difficile capire che tempo usare in queste pagine: scrivo da vivo quindi dovrei usare il futuro, ma tu mi leggerai quando sarò morto, e allora potrei narrare al passato. Spero perciò mi perdonerai qualche errore con i verbi e i tempi, passati, presenti e futuri.

A proposito del futuro, il mio sarà cessato insieme alla vita: questo pensa tuo padre. Ho tentato di fare mio l'insegnamento che l'allievo Platone mette in bocca al suo adorato maestro Socrate: i filosofi "di niente altro si curano se non di morire\footnote{Platone, Fedone.}". Questa citazione è bellissima, carica di significato, però presa nel suo contesto, ovvero il dialogo platonico "Fedone", afferma che la vita deve servire a prepararsi alla morte in vista di quello che verrà successivamente al trapasso; è il testo che ha posto le basi per tutte le speculazioni filosofiche occidentali, cristiane e non cristiane, che vedono l'essenza dell'uomo nella sua anima, nella nostra mente vista come qualcosa di separata dal corpo, dando origine a millenni di visione dualistica mente-corpo. 

Io non credo alla sopravvivenza di nessuna coscienza del proprio sé successiva alla morte, non credo nell'anima separata dai nostri visceri, pronta a liberarsi quando questi cessano di esistere. Ti cito quindi due versi di una canzone, presi da un contesto più ateo e materialista:

\begin{quotation}
	\small La morte è insopportabile per chi non riesce a vivere
	
	La morte è insopportabile per chi non deve vivere\footnote{CCCP - Fedeli alla linea, Brano: Morire, Album: Affinità-divergenze fra il compagno Togliatti e noi. }.
\end{quotation}
