\begin{quotation}
	\small L'uomo libero a nessuna cosa pensa meno che alla morte: e la sua saggezza è una meditazione della vita, non della morte - Baruch Spinoza, Ethica more geometrico demonstata
\end{quotation}

Quando leggerai questo piccolo libro, caro Gino mio, babbo non ci sarà più. E' difficile capire che tempo verbale usare in queste pagine: scrivo da vivo quindi dovrei usare il futuro, ma tu mi leggerai quando sarò morto, e allora potrei narrare al passato. Spero perciò mi perdonerai qualche incongruenza tra tempi passati, presenti e futuri.

A proposito del futuro, il mio sarà cessato insieme alla vita: questo pensa tuo padre. Ho tentato di fare mio l'insegnamento che l'allievo Platone mette in bocca al suo adorato maestro Socrate: i filosofi "di niente altro si curano se non di morire\footnote{Platone, Fedone.}".